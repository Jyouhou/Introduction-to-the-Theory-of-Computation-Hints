\section{Chapter 2: Context-Free Languages}

\begin{itemize}
	
	
	\item[\hard 2.19]
	\begin{comment}
	Let CFG $G$ be the following grammar.
	\[
		\begin{array}{rcl}
			S & \to & \texttt{a}S\texttt{b} \ | \ \texttt{b}Y \ | \ Y\texttt{a} \\
			Y & \to & \texttt{b}Y \ | \ \texttt{a}Y \ | \ \epsilon
		\end{array}
	\]
	Give a simple description of $L(G)$ in English. Use that description to give a CFG for $\overline{L(G)}$, the complement of $L(G)$.
	
	\hint
	\end{comment}
	$Y \to (\tt{a} \cup \tt{b})^*$ and $S \to \tt{a}^n (\tt{b} Y \cup Y \tt{a}) \tt{b}^n$ where $n \in \mathbb{N}$.
	
	
	\item[\hard 2.21] 
	\begin{comment}
	Let $\Sigma = \{ \tt{a}, \tt{b} \}$. Give a CFG generating the language of strings with twice as many $\tt{a}$'s as $\tt{b}$'s. Prove that your grammar is correct.
	
	\hint
	\end{comment}
	Define $\chi : \Sigma^* \to \mathbb{Z}$ by $\chi(x) = n_{\tt{a}}(x) - 2 n_{\tt{b}}(x)$, where $n_{\tt{a}}(x)$ counts the number of $\tt{a}$s in $x$. Suppose $x = x_1 x_2 \dots x_m$ with $x_i \in \Sigma$. What will happen to $x_l x_{l+1} \cdots x_r$ if $\chi(x_1 x_2 \cdots x_r) = \chi(x_1 x_2 \cdots x_{l-1})$?
	
	
	\item[\hard 2.22] 
	\begin{comment}
	Let $C = \{ x \tt{\#} y \ | \ x,y \in \{\tt{0},\tt{1}\}^* \text{ and } x \neq y \}$. Show that $C$ is a context-free language.
	
	\hint
	\end{comment}
	$C = \{ x \tt{\#} y \ | \ |x| \neq |y| \} \cup \bigcup_{i \in \mathbb{Z}^+} \{ x \tt{\#} y \ | \ |x| = |y| \text{ and } x_i \neq y_i \}$, where $x_i$ is denoted as the $i$-th character of $x$.
	
	
	\item[\hard 2.23]
	\begin{comment}
	Let $D = \{ xy \ | \ x,y \in \{\tt{0},\tt{1}\}^* \text{ and } |x|=|y| \text{ but } x \neq y \}$. Show that $D$ is a context-free language.
	
	\hint
	\end{comment}
	Solve problem 2.22 first.
	
	\item[\hard 2.24]
	\begin{comment}
	Let $E = \{ \tt{a}^i \tt{b}^j \ | \ i \neq j \text{ and } 2i \neq j \}$. Show that $E$ is a context-free language.
	
	\hint
	\end{comment}
	$E = \{ \tt{a}^i \tt{b}^j \ | \ j < i \} \cup \{ \tt{a}^i \tt{b}^j \ | \ i < j < 2i \} \cup \{ \tt{a}^i \tt{b}^j \ | \ j > 2i \} $.
	
	
	\item[\hard 2.27]
	\begin{comment}
	\newcommand{\abrsc}[1]{\abr{\sc{#1}}}
	Let $G = (V, \Sigma, R, \abr{\sc{stmt}})$ be the following grammar.
	\[
		\begin{array}{rcl}
			\abrsc{stmt} & \to & \abrsc{assign} \ | \ \abrsc{if-then} \ | \ \abrsc{if-then-else} \\
			\abrsc{if-then} & \to & \tt{if condition then } \abrsc{stmt} \\
			\abrsc{if-then-else} & \to & \tt{if condition then } \abrsc{stmt} \tt{ else } \abrsc{stmt} \\
			\abrsc{assign} & \to & \tt{a:=1}
		\end{array}
	\]
	$G$ is a natural-looking grammar for a fragment of a programming language, but $G$ is ambiguous.
	\begin{itemize}
		\item[a.]
		Show that $G$ is ambiguous.
		\item[b.]
		Give a new unambiguous grammar for the same language.
	\end{itemize}

	\hint
	\end{comment}
	For part b, try to let every $\tt{else}$ correspond to the nearest $\tt{if}$, like what C/C++ grammar specifies.
	
	
	\item[\hard 2.28]
	Solve problem 2.21 first.
	
	
	\item[\hard 2.29]
	Use the pumping lemma.
	
	
	\item[\hard 2.33]
	Use the pumping lemma on $\tt{a}^{(p+1)^2} \tt{b}^{p+1}$.
	
	
	\item[\hard 2.37]
	$R$'s appearing twice gives us the pumping lemma for CFL. What if it appears thrice?
	
	
	\item[\hard 2.40]
	Use the pumping lemma.
	
	
\end{itemize}